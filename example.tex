%!TEX program = xelatex
\documentclass[garamond,namecite]{goose-article}

\usepackage{lipsum,verbatim,mdframed}

\title{%
  goose-article: customized \LaTeX-article
}

\author[1]{T.W.J.~de~Geus$^{*,}$}

\affil[1]{
  Department of Mechanical Engineering \nl
  Eindhoven University of Technology \nl
  The Netherlands
}

\contact{%
  $^*$Contact: %
  \href{mailto:tom@geus.me}{tom@geus.me} %
  \hspace{1mm}--\hspace{1mm} %
  \href{http://www.geus.me}{www.geus.me}%
}

\hypersetup{pdfauthor={T.W.J. de Geus}}

\header{%
  goose-article
}

% %%%%%%%%%%%%%%%%%%%%%%%%%%%%%%%%%%%%%%%%%%%%%%%%%%%%%%%%%%%%%%%%%%%%%%%%%%%%%%
\begin{document}
% %%%%%%%%%%%%%%%%%%%%%%%%%%%%%%%%%%%%%%%%%%%%%%%%%%%%%%%%%%%%%%%%%%%%%%%%%%%%%%

\maketitle

\begin{abstract}
\texttt{goose-article} is a customized class designed for scientific articles. The usage is similar to the default \texttt{article}-class while the class takes care of formatting.
\end{abstract}

\keywords{\LaTeX; class; article}

% ==============================================================================
\section{Preamble}
% ==============================================================================

% ==============================================================================
\subsection{Introduction}
% ==============================================================================

By default most of the standard \LaTeX-packages are loaded. Any of these packages can be re-loaded, with other defaults, without problems. In addition the title, the authors and their affiliations, contact information, and optionally a header should be specified; see below.

% ==============================================================================
\subsection{Load class}
% ==============================================================================

To load the class use
\begin{verbatim}
  \documentclass{goose-article}
\end{verbatim}
%
To use customized fonts, the documents has to be compiled using XeLaTeX. For example:
\begin{verbatim}
  %!TEX program = XeLaTeX
  \documentclass[garamond]{goose-article}
\end{verbatim}
%
The following fonts are available:
%
\begin{itemize}
  %
  \item \texttt{garamond}
  \item \texttt{times}
  \item \texttt{verdana}
  %
\end{itemize}
%
%
Furthermore the following options are available
%
\begin{itemize}
  %
  \item \texttt{narrow}: widen the margins of the page, useful during the review process;
  \item \texttt{doublespacing}: set the line-spacing to double, useful during the review process.
  %
\end{itemize}
%

% ==============================================================================
\subsection{Title, authors, and headers}
% ==============================================================================

%
\begin{itemize}
%
\item The \textit{title} is specified using
\begin{verbatim}
  \title{...}
\end{verbatim}
%
\item The \textit{author(s)} and their \textit{affiliation(s)} are formatted using the \texttt{authblk}-package. The interface of this package is retained. Basically there are two ways to specify authors and affiliations. If there is a single affiliation:
\begin{verbatim}
  \author{...}
  \author{...}
  \affil{...}
\end{verbatim}
To account for multiple affiliations, identifiers such as number can be used:
\begin{verbatim}
  \author[1]{...}
  \author[1,2]{...}
  \affil[1]{...}
  \affil[1,2]{...}
\end{verbatim}
%
Note that a new line can be forced by using \verb|\nl|. The default \verb|\\| does not work in the \texttt{authblk}-package.
%
\item Contact information is displayed below the affiliations using
\begin{verbatim}
  \contact{...}
\end{verbatim}
%
\item The upper-header (opposite to the page number) can be specified using
\begin{verbatim}
  \header{...}
\end{verbatim}
%
\item Additionally one could decide to change the author of the PDF-document
\begin{verbatim}
  \hypersetup{pdfauthor={...}}
\end{verbatim}
%
\end{itemize}
%

% ==============================================================================
\section{Document layout}
% ==============================================================================

The basic document layout is as follows

\begin{mdframed}
\begin{verbatim}
%!TEX program = xelatex
\documentclass[options]{goose-article}

\title{...}

\author{...}

...

\begin{document}

  \maketitle

  \begin{abstract}
  ...
  \end{abstract}

  \keywords{...}

  ...

\end{document}
\end{verbatim}
\end{mdframed}
which is pretty self-explanatory.

% ==============================================================================
\section{Citations}
% ==============================================================================

Citations and references are handled using \texttt{natbib}. To cite use
\begin{verbatim}
  \citep{...}  (or \cite{...})
  \citet{...}
\end{verbatim}
The former only inserted a citation as number. For example \citep{geus}. The latter also includes the name(s) of the author(s). For example \citet{geus}.

The bibliography information is stored in a \texttt{bib}-file, which is included using
\begin{verbatim}
  \bibliography{...}
\end{verbatim}
This command creates a section ``References'' with the bibliography in order of appearance.

Note that a large part of the formatting of Bib\TeX~depends on the formatting of the \texttt{bib}-file. A Python-script \texttt{bibparse} is available to automatically clean-up the formatting of the \texttt{bib}-file. An updated \texttt{unsrtnat.bst} is available that includes the \texttt{eprint} field.

% ==============================================================================
\bibliography{example_refs}
% ==============================================================================

% %%%%%%%%%%%%%%%%%%%%%%%%%%%%%%%%%%%%%%%%%%%%%%%%%%%%%%%%%%%%%%%%%%%%%%%%%%%%%%
\end{document}
% %%%%%%%%%%%%%%%%%%%%%%%%%%%%%%%%%%%%%%%%%%%%%%%%%%%%%%%%%%%%%%%%%%%%%%%%%%%%%%


